\documentclass[a4paper,10pt,oneside]{article}
\setlength{\columnsep}{15pt}    %兩欄模式的間距
\setlength{\columnseprule}{0pt}

\usepackage[landscape]{geometry}
\usepackage{amsthm}								%定義,例題
\usepackage{amssymb}
\usepackage{fontspec}								%設定字體
\usepackage{color}
\usepackage[x11names]{xcolor}
\usepackage{xeCJK}								%xeCJK
\usepackage{listings}								%顯示code用的
%\usepackage[Glenn]{fncychap}						%排版,頁面模板
\usepackage{fancyhdr}								%設定頁首頁尾
\usepackage{graphicx}								%Graphic
\usepackage{enumerate}
\usepackage{titlesec}
\usepackage{amsmath}
\usepackage{pdfpages}
\usepackage{multicol}
\usepackage{fancyhdr}
%\usepackage[T1]{fontenc}
\usepackage{amsmath, courier, listings, fancyhdr, graphicx}

%\topmargin=0pt
%\headsep=5pt
\textheight=530pt
%\footskip=0pt
\voffset=-20pt
\textwidth=800pt
%\marginparsep=0pt
%\marginparwidth=0pt
%\marginparpush=0pt
%\oddsidemargin=0pt
%\evensidemargin=0pt
\hoffset=-100pt

\setmainfont{Courier}				%主要字型
\setCJKmainfont{PingFangTC-Regular}			%中文字型
% \setmainfont{Linux Libertine G}
\setmonofont{Courier}
% \setmainfont{sourcecoepro}
\XeTeXlinebreaklocale "zh"						%中文自動換行
\XeTeXlinebreakskip = 0pt plus 1pt				%設定段落之間的距離
\setcounter{secnumdepth}{3}						%目錄顯示第三層

\makeatletter
\lst@CCPutMacro\lst@ProcessOther {"2D}{\lst@ttfamily{-{}}{-{}}}
\@empty\z@\@empty
\makeatother
\lstset{											% Code顯示
language=C++,										% the language of the code
basicstyle=\scriptsize\ttfamily, 						% the size of the fonts that are used for the code
numbers=left,										% where to put the line-numbers
numberstyle=\tiny,						% the size of the fonts that are used for the line-numbers
stepnumber=1,										% the step between two line-numbers. If it's 1, each line  will be numbered
numbersep=5pt,										% how far the line-numbers are from the code
backgroundcolor=\color{white},					% choose the background color. You must add \usepackage{color}
showspaces=false,									% show spaces adding particular underscores
showstringspaces=false,							% underline spaces within strings
showtabs=false,									% show tabs within strings adding particular underscores
frame=false,											% adds a frame around the code
tabsize=2,											% sets default tabsize to 2 spaces
captionpos=b,										% sets the caption-position to bottom
breaklines=true,									% sets automatic line breaking
breakatwhitespace=false,							% sets if automatic breaks should only happen at whitespace
escapeinside={\%*}{*)},							% if you want to add a comment within your code
morekeywords={*},									% if you want to add more keywords to the set
keywordstyle=\bfseries\color{Blue1},
commentstyle=\itshape\color{Red4},
stringstyle=\itshape\color{Green4},
}


\newcommand{\includecpp}[2]{
  \subsection{#1}
    \lstinputlisting{#2}
}

\newcommand{\includetex}[2]{
  \subsection{#1}
    \input{#2}
}


\begin{document}
  \begin{multicols}{3}
  \pagestyle{fancy}

  \fancyfoot{}
  \fancyhead[L]{NYCU-Segmentree}
  \fancyhead[R]{\thepage}

  \renewcommand{\headrulewidth}{0.4pt}
  \renewcommand{\contentsname}{Contents}


  \scriptsize
  \section{Dp}
  \includecpp{01\_knapsack}{./DP/01_knapsack.cpp}
  \includecpp{Josephus}{./DP/Josephus.cpp}
  \includecpp{Bitmask}{./DP/Bitmask.cpp}
  \includecpp{InfinitKnapsack}{./DP/InfinitKnapsack.cpp}
\section{Data Structure}
  \includecpp{DSU}{./Data Structure/DSU.cpp}
  \includecpp{Monotonic Queue}{./Data Structure/Monotonic Queue.cpp}
  \includecpp{BIT}{./Data Structure/BIT.cpp}
  \includecpp{Treap}{./Data Structure/Treap.cpp}
  \includecpp{Segment Tree}{./Data Structure/Segment Tree.cpp}
  \includecpp{Sparse Table}{./Data Structure/Sparse Table.cpp}
  \includecpp{Monotonic Stack}{./Data Structure/Monotonic Stack.cpp}
\section{Flow}
  \includecpp{Maximum Simple Graph Matching}{./Flow/Maximum Simple Graph Matching.cpp}
  \includecpp{Dinic}{./Flow/Dinic.cpp}
\section{Formula}
  \includetex{formula}{./Formula/formula.tex}
\section{Geometry}
  \includecpp{Sort by Angle}{./Geometry/Sort by Angle.cpp}
  \includecpp{Geometry}{./Geometry/Geometry.cpp}
  \includecpp{Convex Hull}{./Geometry/Convex Hull.cpp}
  \includecpp{Min Covering Circle}{./Geometry/Min Covering Circle.cpp}
  \includecpp{Point in Polygon}{./Geometry/Point in Polygon.cpp}
\section{Graph}
  \includecpp{Bipartite Matching}{./Graph/Bipartite Matching.cpp}
  \includecpp{Tarjan SCC}{./Graph/Tarjan SCC.cpp}
  \includecpp{Bridge}{./Graph/Bridge.cpp}
  \includecpp{2 SAT}{./Graph/2 SAT.cpp}
  \includecpp{Kosaraju 2DFS}{./Graph/Kosaraju 2DFS.cpp}
  \includecpp{Minimum Steiner Tree}{./Graph/Minimum Steiner Tree.cpp}
  \includecpp{Dijkstra}{./Graph/Dijkstra.cpp}
  \includecpp{Maximum Clique Dyn}{./Graph/Maximum Clique Dyn.cpp}
  \includecpp{Minimum Clique Cover}{./Graph/Minimum Clique Cover.cpp}
  \includecpp{Floyd Warshall}{./Graph/Floyd Warshall.cpp}
  \includecpp{Articulation Vertex}{./Graph/Articulation Vertex.cpp}
  \includecpp{Number of Maximal Clique}{./Graph/Number of Maximal Clique.cpp}
  \includecpp{DominatorTree}{./Graph/DominatorTree.cpp}
  \includecpp{Topological Sort}{./Graph/Topological Sort.cpp}
  \includecpp{Closest Pair}{./Graph/Closest Pair.cpp}
  \includecpp{Minimum Mean Cycle}{./Graph/Minimum Mean Cycle.cpp}
  \includecpp{Planar}{./Graph/Planar.cpp}
  \includecpp{Heavy Light Decomposition}{./Graph/Heavy Light Decomposition.cpp}
  \includecpp{Centroid Decomposition}{./Graph/Centroid Decomposition.cpp}
  \includecpp{KM\_O}{./Graph/KM_O.cpp}
  \includecpp{Minimum Arborescence}{./Graph/Minimum Arborescence.cpp}
  \includecpp{Maximum Clique}{./Graph/Maximum Clique.cpp}
\section{Math}
  \includecpp{Miller Robin}{./Math/Miller Robin.cpp}
  \includecpp{fpow}{./Math/fpow.cpp}
  \includecpp{modinv}{./Math/modinv.cpp}
  \includecpp{PollardRho}{./Math/PollardRho.cpp}
  \includecpp{Euler Totient Function}{./Math/Euler Totient Function.cpp}
  \includecpp{extGCD}{./Math/extGCD.cpp}
  \includecpp{random}{./Math/random.cpp}
  \includecpp{FFT}{./Math/FFT.cpp}
  \includecpp{mu}{./Math/mu.cpp}
\section{Misc}
  \includecpp{Mo's Algorithm}{./Misc/Mo's Algorithm.cpp}
  \includecpp{pbds}{./Misc/pbds.cpp}
  \includecpp{Misc}{./Misc/Misc.cpp}
  \includecpp{bit}{./Misc/bit.cpp}
\section{String}
  \includecpp{Hashing}{./String/Hashing.cpp}
  \includecpp{Trie}{./String/Trie.cpp}
  \includecpp{Zvalue}{./String/Zvalue.cpp}
  \includecpp{KMP}{./String/KMP.cpp}
  \includecpp{Manacher}{./String/Manacher.cpp}
\section{Tree}
  \includecpp{LCA}{./Tree/LCA.cpp}
  \includecpp{Diameter}{./Tree/Diameter.cpp}
  \includecpp{Radius}{./Tree/Radius.cpp}
  \includecpp{Spanning Tree}{./Tree/Spanning Tree.cpp}

  \clearpage
  \end{multicols}
  \newpage
  \begin{multicols}{3}
  \enlargethispage*{\baselineskip}
  \begin{center}
    \Huge\textsc{NYCU\_Segmentree Codebook}
    \vspace{0.35cm}
  \end{center}
  \tableofcontents
  \end{multicols}
  \clearpage
\end{document}